\documentclass{article}

\usepackage{linguex}

\usepackage{geometry}

\geometry[10pt,letterpaper]{vmargin=1in,hmargin=1in}

\begin{document}
  
\section{Correctives}

names for corrective expressions/utterances/contributions: correctives
names for the resiliant assertions that allow for correction: correctibles, posits, leaps (from james)


Plurals are functions from sets (of individuals or pluralities) to partitions over those sets.  The uniformity (homogeneity) presupposition gets cashed out as a contextually determined default (null) partition.  

Compare:

\ex. \a. Bears are dangerous.
\b. [Grizzley]$_f$ bears are.

\ex. \a. All bears are dangerous.
\b. [Grizzley]$_f$ bears are.

The contextualism/relativism debate often revolves around the resiliance to correction of various assertions.

\ex. \a. I know $\alpha$.
\b. What about the possibility that you are a $BIV$?
\a. I didn't mean I would be able to convince a skeptic!

The debate in the literature as well as the data examined in ``CIA Leaks'' suggest that there is variability in the corrective potential of certain assertions. At least, corrections of this sort are different than flat denials.

This resiliancy is also a defining characteristic of generics:

\ex. \a. Cows give milk.
\b. What about bulls?
\a. I didn't mean $all$ cows!

I have a proposal, in the spirit of relativism, that can account for the distinctive corrective pattern for these expressions. The idea is that the semantic values of such expressions are not sets (or properties) within the relevant domain.  Instead, the semantic values are functions from sets (properties) to partitions (relations) thereof.

\[c \rightarrow f(s) \rightarrow R(s,s)\]

\[f(s,c) \rightarrow R(s,s)\]

In general, assertions invoke a null partition, thus presupposing uniformity of the set with respect to the semantic predicate of the assertion.  However, contrastive topic in a response to such an assertion indicates a denial of the proposed partitioning, without denying the whole assertion.

Corrections form a diverse group, but one element they share is that they seem to constitute a denial of something \textit{smaller} than the originally asserted content.  This may be a presupposition of the sentence or one of its contained expressions, a syntactic element of the sentence, or even the perceived tone of the utterance.  That corrections deny ony a part of the assertion (utterance) is important because some elements make their way into the common ground for future reference.  If they were straight denials, we would not expect an update to the common ground.

Consider:

\ex. \a. The bear is hungry.
\b. No, it's not.

The denial allows the presupposition of ``the bear'' through.

\ex. He didn't call the po-lis; he called the po-lis.

The denial accepts the content, but blocks a sociological implicature associated with dialect (cadence, pronunciation, emphasis).

\ex. \a. Bears are dangerous.
\b. [Grizzley]$_f$ bears are.

The correction involves a denial, but it doesn't halt the conversation. Instead, it let's something through and immediately elaborates on it. What is it that is carried through? Perhaps, it can be cashed out in terms of the primary QUD or the interrogative update.  The idea is that the correction makes both a negative and a positive contribution to the discourse, and the positive contribution has to work on something that's already been introduced.  What it works on is the set that is mentioned by the first speaker, but it's contribution is to invoke a different partitioning of that set.  This is what sets corrections apart from a mere conjunction of a denial and a new contribution.  The denial is merely of the default partition; the contribution continues off of the original assertion.

We'll need a story of how context fixes a partition.  As usual, it will be messy.  But demonstratives look to context for a salient individual, and modals look to context for a set of propositions.  I'm thinking that disputative potential appeals to context for an assortoric function.  Denial resiliant utterances are like hedges and indirect evidentials in that the involve a lesser commitment to the content of the utterance.  This suggests that they are mood related.

So, take the following just so story: The speaker wants to introduce a line of inquiry and advance along that line in one move. But she uses a correctible to allow that the positive contribution may not be accepted. If this sort of move is mood related, then it suggests that there is a different kind of update at play.  Assertions restrict context sets, questions partition context sets, commands rank elements within context sets, and correctibles DO SOMETHING ELSE.

Of course, it really looks like we can just decompose a corrections into a partition and an assertion.  But the assertion part doesn't seem full fledged in an important sense.

The logical aparatus we bring to bear on the puzzle might just be that of partitioning and restricting,  This doesn't mean that corrections needen't be viewed as sui generis contributions. Openendedness, resiliance, lesser commitment, and not-quite assertion seems like unique properties of this sort of contribution.

The difference in the logical aperatus could come not in the kind of update itself, but it how the update is incorporated in a joint, dynamic scoreboard.  We have something like tentative proposals that allow for rejection without eliminating thier viability. (There's an analogy here to Murray's attempts to treat at-issue content as a proposed, as opposed to direct, restriction of the common ground.  Where she ran into trouble was in conjoined evidential sentences.  The mood and content features of the sentences got co-mingled in an unsavory way.)  This is also where the similarity to a relativism of Egan's sort comes in.  Relativism focuses on the fact that there are lots of ways in which our beliefs are similar even though our contexts differ.  This similarity is a matter of an involvement with the content.  Relativists build that involvement into the content of the embedding attitude.  But this gets unsavory in contexts where we want to allow involvement to vary.

We may get some insight from seeing what type correctible sentences end up being.  If plurals are functions from sets to sets of sets, then they are <et,ett> and we're taking the default null partition to be relative to the semantic predicate, so it can be just a regular set <e,t>.  So, we need a composition rule that can take these elements and make an assertible content out of it.  Or we need a pseudo-assertible kind of content.

\paragraph{Option 1:} Subject contributes an open partition over the set denoted, null by default.

\paragraph{Option 2:} The open partition is over the whole domain, with the subject indicating one cell of the suggested partition.

Maybe correctibles are assertions with less force or import.  As how knowledge is one end of a scale of justified belief, full assertion is an end where the norm is knowledge. One can employ assertive force in a contribution without claiming knowledge if it is good enough to advance the current conversation.  Compare the use of wikipedia in common discourse.
\end{document}
