\documentclass[letterpaper,10pt]{article}

\usepackage[colorlinks=true,linkcolor=blue]{hyperref}

%Symbol packages
\usepackage{amssymb,mathabx,stmaryrd}

%Margin settings
\usepackage{geometry}
\geometry{hmargin={.75in,.75in},vmargin={1in,1in}}

%Header settings
\usepackage{fancyhdr}
%\setlength{\headheight}{15pt}
\pagestyle{fancy}
\fancyhead{}
\fancyhead[L]{Erik Hoversten}
\fancyhead[C]{}
\fancyhead[R]{Disagreement in content}

%Paragraph settings
\setlength{\parindent}{0pt}
\setlength{\parskip}{2ex plus .5ex minus .2ex}

%numbering and reference settings
\usepackage{hyperref}
\usepackage{linguex}
\usepackage{enumitem}

\begin{document}

\section{Contrastive focus}

Consider the following discourse, which involves an instance of what has come to be called \textit{contrastive focus}.

\ex. \a. What do bears eat?
\b. [Grizzly bears]$_{cf}$ eat [salmon]$_{f}$.

English presents a general pattern of placing intonational stress on the element of a sentence that represents the new information in the discourse.  This is borne out in the stress on the topic of sentence \Last[b], \textit{salmon}, which I have represented with a subscripted $f$.  But what is not new is given, and that is standardly unstressed.  Our example violates that pattern in placing a second element of stress within the semantic comment of sentence \Last[b], represented with a subscripted $cf$.

The standard way to analyze members of the family of focus related interpretive effects is to treat them as presuppositional.  Focused elements are associated with a focus semantic value, and utterances involving focus are felicitous so long as an appropriate antecedent of that value can be found in prior discourse. So, for instance, \textit{salmon} in our example discourse is associated with a  set of alternatives, which, appropriately contextually restricted, may look something like $\{ants,\;berries,\;cats,\;dandelions,\ldots\}$.  Questions, for their part, are semantically associated with sets as well; namely, the set of possible answers to the question.  Given appropriate contextual restriction, it is plausible that these two sets will match, or at least significantly overlap, and our antecedent has been located.  \textit{Grizzly bears}, on the other hand, has no obvious antecedent in the discourse, which makes the felicity of utterance \Last[b] mysterious.

It's fruitful to think of discourses as taking place within a framework of goals, and to assess contributions to the discourse in part by how far they take us along the path toward satisfying those goals. In an important sense, questions are vehicles for adding goals to the discourse structure.  They set up a constrained set of alternatives and call for information that would select one from among that set. Pushing this idea along, we may suggest that discourse goals are fully determined by questions, either explicitly posed, or implicity accepted.  And, questions can be embedded one within another, where subquestions call for answers that provide some but not all of the information needed to achieve the goal set out by their superquestions. Assuming this more rich discourse structure, we have more resources to locate an antecedent for our contrastive focus.

\Last[a] introduces a discourse goal, and the maximally helpful interlocutor would provide a complete answer thereby achieving the goal.  We can't all be maximally helpful all the time, but that doesn't mean we can't still contribute to the process.  For, questions allow for partial answers, in the form of complete answers to subquestions embedded within them.  \Last[b] contributes to acheiving the goal set by \Last[a] by giving a complete answer to one of these subquestions.  Thus, we can think of \Last[a] as adding a richer, structured goal to the discourse.

\begin{itemize}[label={}]
  \item What do bears eat?
  \begin{itemize}[label={$\drsh$}]
    \item What do grizzly bears eat?
    \item What do black bears eat?
    \item What do polar bears eat?
  \end{itemize}
\end{itemize}

The felicity of contrastive focus, then, tracks the presence in the discourse goal structure of a subquestion to which the utterance provides a complete answer.  The presuppostional account does raise some interesting questions.  Why think that the list of subquestions presented above is the correct way of representing the discourse structure at the time of utterance of \Last[b].  After all, \Next could serve as an equally felicitous response to \Last[a], but it answers a question that presumably would not have been among the intended list of subquestions.

\ex. [Hungry bears]$_{cf}$ will eat anything.

The immediate response to this challenge is to point out that many presuppostions can go through in conversations in which their antecedents haven't been explicitly introduced.  Some mechanism of accommodation of these presuppositions will have to play a role in any theory of discourse pragmatics, and we can appeal to the same mechanism here.  Perhaps discourse structure is not so rich prior to the utterance of \Last[b], but the utterance creates the structure it needs.  And perhaps a secondary feature of the use of intonational stress to implement contrastive focus is to signal the need for this divergence from the antecedently accepted discourse structure.

Appeal to accommodation, though, is not much of a theory on its own, and even if presuppositions are easily accommodated, part of their theoretical utility stems from the fact that they admit of presupposition failure.  But contrastive focus is remarkably unconstrained.  Even for unusual continuations of the discourse, it is difficult to maintain that they are infelicitous.

\ex. \a. What do bears eat?
\b. [Hungry grizzly bears with cubs]$_{cf}$ would eat a [person]$_f$.
\c. \#Hey, wait a minute. I didn't know there were such kinds of bear.

Perhaps a presupposition that never fails is no presupposition at all, but I wouldn't want to hang my hat on that proposed necessary condition.  But now consider the following discourse, which involves an intonationally similar use of focus.

\ex. \a. Bears are dangerous.
\b. [Grizzly bears]$_{cf}$ are dangerous.

In this example, there is no explicit question available at all. We could perhaps construct an implicit discourse goal structure, and derive an antecedent from there, but the route is not at all clear.

Call this use of focus, tendentiously, \textit{corrective focus}.  I want to maintain that this use of focus is not entirely presuppositional, in that it makes a forward directed contribution to the discourse structure.

The speaker of a contrastive focus does two things.  First, she denies an element of the first speaker's utterance.  Contrastive focus has a disputative element.  Second, she offers a positive proposal, which differs from the original speaker's offering, but is systematically related to it.  Sentences involving contrastive focus provide a \textit{correction} of an element of preceding discourse.

Corrections come in many varieties.  In the example above, the correction restricts the class of individuals the predicate is said to apply to.  But in the next case, the correction expands the class.

\ex. \a. I love bacon.
\b. [Everybody]$_{cf}$ loves bacon.

And corrections can also apply to the predicate applied, perhaps by bumping it up or down an implicitly recognized scale.

\ex. \a. Big Papi is a good baseball player.
\b. Big Papi is an [amazing]$_{cf}$ baseball player.

Given the ubiquity of contexts in which focus can be utilized to invoke a correction, why suppose that there is any fundamental unifying interpretive feature involved?  After all, the use of focus to invoke scales, like the Big Papi example, already have satisfactory pragmatic accounts.  It may seem that accounts could similarly be given to the other examples, if perhaps in a piecemeal fashion.

But I think there is a systematic relationship among these uses of focus, and that they will submit to a uniform analysis.  But the analysis requires a conceptual change in what falls within the purview of semantic theory.  

The presupposition (or whatever it is) can't simply be a property associated with the predicate and saturated by the plurality that holds of a class just in case that class is homogeneous with regard to the predicate.  For then, the following could be said to disagree, because they both saturate the same property in a different way.

\ex. \a. Bears are dangerous.
\b. [Guns]$_{cf}$ are dangerous.

Notice that, interestingly, the following doesn't seem so bad if we imagine that \Next[a] is in response to a question ``What animals are dangerous?''

\ex. \a. [Bears]$_f$ are dangerous.
\b. [Sharks]$_{cf}$ are dangerous.

Instead, what we need is some object that captures that the first sentence says \textit{of the class of bears} that it is homogeneous with respect to the predicate (or class of predicates determined by the predicate), and the correction denies this by claiming there is a specific partition over the class of bears.  That is, we need something like:

\ex. \a. $\{\langle x,y\rangle: x,y\in \llbracket bears\rrbracket \;\&\; \forall P\in \{Q: Q = \llbracket dangerous\rrbracket \vee Q=R\}(Px \leftrightarrow Py)\}$
\b. $\{\langle x,y\rangle: x,y\in \llbracket bears\rrbracket\cap \llbracket grizzly\rrbracket \;\&\; \forall P\in \{Q: Q = \llbracket dangerous\rrbracket \vee Q=R\}(Px \leftrightarrow Py)\}$

Contrastive focus is associated with two mutually supporting interpretive effects.  The first is \textit{disputative}.  There is a denial of some element of the first speaker's utterance. This denial needs to track something from the first speaker's utterance, whether that be at-issue content or an implicature, presupposition, or some other form of not-at-issue content.  THe second effect is a correction.  THere is a replacement of the first speaker's contribution with a distinct but related offering.  The relation may be one of subset, superset, or movement along a scale.  But representing the relation seems also to require expanding our understanding of the content of the first speaker's utterance.  Call this expansion one of capturing the \textit{corrective potential} of the utterance.  I think that corrective potential is a criterion against which we can measure the relevance or informativeness of an utterance.

It seems that a fruitful way to represent the dual elements of contrastive focus would be as something like a test in dynamic semantics.  A test is a condition that checks to see that the input has a particular global property and returns the input when passed and returns the null item when failed.  We want our corrective condition to be test-like because we have a presupposition-like effect, and presuppositions can be represented as tests.  But we don't want complete collapse of the output when the condition fails.  Instead, what we want is a structuring condition, in that the whole set is passed through unchanged if the utterance is not challenged, but if it is challenged, it gets structured in a particular way by the challenging utterance.  A key requirement is that the kind of thing picked out by the condition is a thing that can be structured in different ways while retaining its identity.  This is the idea behind using relativist-style properties to represent the condition.

When properties are harnessed to do the work standardly given to propositions, we must bring in backup to saturate the property.  For instance, if properties are the objects of belief, then we have to reinterpret belief as self-attribution.  So the individual holding the belief saturates the property that is the content of it.  Similarly, if plural subjects are associated with a property-type presupposition, the property needs to be saturated in some way.  A common idea here is to posit a \textit{judge} as an element of the circumstance of evaluation for sentences, and let the identity of the judge be determined contextually.  If the judge saturates the property, then we could say that the judge's QUD list partitions the plurality.  This does seem to be an improvement over positing massive accommodation.  

But I'm not fully satisfied with it.  I would much rather have the content of the response sentence directly partition the plurality instead of going through a judge middleman.  The nature of the predicate may have a role in determining the kinds of partition available.



\section{Disagreement}

\subsection{Primer}
Disagreement has recently been invoked as a test case in evaluating semantic approaches to expressions such as predicates of personal taste, epistemic modals, and moral evaluations.  The main concern has been whether a particular account of the semantics of those expressions is compatible with cases of intuitive disagreement utilizing those expressions.

Corrective focus also has a disputative element to it, though it isn't easily traced to any particular expression of the sentence in which it appears.  Still, the insights from the contextualist/relativist debates over disagreement may be relevant here.

There are two distinct sets of combatants in battles billed as ``Contextualist v. Relativist''.  The first takes the issue to center on the character/content distinction familiar from Kaplan (1989) and Lewis (1991).  Context plays two roles in the determination of truth values for sentences.  First, a sentence has a character, which is a (perhaps constant) function from contexts onto contents.  Second, contents are themselves functions from circumstances of evaluation onto truth values.  On one characterization of the debate, contextualists maintain that the expressions under consideration depend on context only in its content determining role.  Relativists maintain that contextual features such as the identification of a judge for evaluating the content may be relevant to determining that content's truth value.

On the other characterization of the debate, the decision point not comes at which level context comes in, but which context is relevant at that level.  Contextualists maintain that all context sensitivity is derived from elements of the context surrounding the utterance.  Relativists hold that contexts differing in interesting ways from that of the speech environment may be relevant to determining truth.  For instance, it may be that an individual at a location distant from the speech event who hears what is spoken rightfully invests his interpretation of it with elements of his own environment.

\subsection{Enrichment}
But this isn't the only way relativist/contextualist debates can flow.  First, disputation is widespread, and it need not be limited to semantically generated content.  This fact gives new avenues for contextualists to explore in search of their lost disagreement.  Second, not all disagreement materializes as flat rejection of a proposal.  In fact, in its most frequent guise, disagreement is accompanied by a new contribution that either explains the source of or reason for the disagreement or provides an alternative to the original proposal for which there is no disagreement. Disagreement needs an atencedent, but it need not throw that antecedent out entirely.

\begin{enumerate}
  \item Disagreement at many different levels of interp.
  \item Some reason based motivations for including disagreement within meaning representation broadly construed.
  \item A need to say how it is represented. Both theory neutral and in a theory specific way.
  \item Elements of dynamic semantics provide a useful way of implementing an intuitive understanding, unifying principles, and incorporating ideas from other debates.
  \item Though, it may be ambitious to suggest we've captured all of linguistic disagreement.
\end{enumerate}
  
Relativists propose to capture disagreement by way of "open content". Cf Evans, Richard on temporally neutral props.
The similarity presupp? Properties require saturation, or closing of the content, which can be done in different ways.
A relativist implementation that doesn't screw with plausible uptake conditions.
Conditional relativity, to put forward the strongest claim one can get away with, while being ready to back off it challenged.

\section{The case of \textit{however}}
The notion of corrective content is not just relevant to the general linguistic phenomenon of contrastive focus.  It also seems to be realized lexically in English by the connective, ``however''.

However can occur in different locations within a sentence.  The position may serve to focus on a particular element, but the denial seems to be of the entire proposition no matter it's location.

\ex. \a. $X$, $Y$ \\ However, $Z$.
\b. $x$, however, is $y$.

\ex. \a. Birds fly. \\ However, kiwi birds are flightless.
\b. Most birds fly. \\ However, kiwi birds are flightless.

examples of multisentential relations

\begin{enumerate}
  \item \textit{Therefore} is an entailment particle, which related contents of two sentences in discourse.
  \item Assertion is a discourse maneuver that related a scoreboard to the content of a sentence asserted.
  \item \textit{If} is a relation between contents (relative to their respective contexts).
  \item Disagreement is a discourse maneuver that relates two contents, or a content to the scoreboard (for presupposition denial).
  \item Contrastive focus is a discourse particle that carries disputative content.
  \item \textit{However} is a denial particle, which relates the contents of two sentences in discourse.
\end{enumerate}

Brasoveanu on \textit{therefore}.  \textit{Therefore}, like \textit{if} and \textit{must}, introduces a propositional dref, and is a relation between propositions, relative to a modal base and an ordering source.  For logical entailment (\textit{therefore} can capture different entailment relations), both the modal base and ordering source are empty.  Thus, the restrictor is treated as identical to the scope.  \textit{Therefore} has an anaphoric element that requires a premise proposition to which to relate its scope.  So, Brasoveanu posits a covert content formation particle ($if^{p_1}$) that scopes over the premise and stores its content for later retrieval.

I suspect that \textit{however} and contrastive focus are doing something similar.

\end{document}
