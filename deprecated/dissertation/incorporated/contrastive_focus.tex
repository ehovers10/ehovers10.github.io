\documentclass[10pt]{article}
\usepackage[utf8]{inputenc}

\usepackage{geometry}
\geometry{hmargin=1in,vmargin=1in}

\usepackage{linguex}

%opening
\title{Depth of focus}
\author{Erik Hoversten}

\begin{document}

\maketitle

\section{The varieties of focus related interpretational effects}

Focus as information structuring device in natural language is implicated in a number of interpretational effects ranging from the merely pragmatic to the strongly semantic.  These effects include:

\begin{itemize}
 \item Marking the distinction between \textbf{given} and \textbf{new} information.
 \item Determining felicity of responses to preceding questions in discourse.
 \item Determining truth conditions for sentences containing focus-sensitive particles such as \textit{even} and \textit{only}.
\end{itemize}

The interpretational effects are relatively uncontroversial, but the proposed underlying mechanisms that explain these effects are numerous.  Some of the dimensions of variance among theoretical proposals are:
\begin{itemize}
 \item Is the association with focus seen in such particles as \textit{even} and \textit{only} written directly into the lexical semantics for these expressions, or is it the result of a more general pragmatic interaction of focus with interpretation?
 \item Is the information structuring evident in natural language communication simply a matter of dividing information into a given component and a new component, or should a discourse be viewed as a highly structured entity with a series of nested question and answer pairs?
 \item Are the interpretational effects of focus purely presuppositional?
\end{itemize}

Two of the more interesting interpretational effects of focus arise from what have been called \textit{second occurrence focus} and \textit{contrastive focus} respectively.  Second occurrence focus characterized by a discourse such as \Next.

\ex. \a. Sara only gave cookies to [the second graders]$_F$.
\b. No, [Mike]$_F$ only gave cookies to the second graders.

The key feature of this exchange is that \textit{the second graders} has been defocused in \Last[b] despite the presence of the focus sensitive operator \textit{only}. Contrastive focus is characterized by the following.

\ex. \a. Who showed up to the conference?
\b. [Sally]$_F$ came for the [first day]$_F$.

Common ground in the debate over focus interpretation is that focusing of elements in a sentence is licensed by the presence of a suitable antecedent in previous discourse.  Thus, the focus on \textit{Sally} is licensed by the \textit{who} in the preceding question.  In this example, the focus on \textit{first day} does not have an overt antecedent from which to derive its felicity.  Instead, the focus seems to be pointing to an unrecognized set of contrasting elements subsuming the denotation of \textit{the conference}.

One way to evade the puzzle posed by contrastive focus is to treat contrastive focus as a distinct kind of information structuring device, which is subject to different felicity conditions, and gives rise to different interpretational effects than standard focusing. (Gundel says that contrastive focus is a linguistic realization of a more general attentional feature of human representation; thus, it is distinctly pragmatic.)  The tripartite structure approach to focus interpretation (I include the Prauge school (Partee) and the event-semantics theorists (Herburger)) also takes contrastive focus to be a distinctive feature of the information conveyed by a sentence.  

Another proposal is to point out that the felicity of \Last[b] in the context of the surrounding discourse is only puzzling if we assume a rather minimal amount of information structuring.  If information structure is limited to distinguishing the given from the new information presented by a sentence, and if the felicity of new information requires a suitable antecedent, then the newness of \textit{first day} is without a proper antecedent.  If, however, we expand our notion of information structure, we may be able to find a suitable antecedent after all.

The plausibility of such an approach to contrastive focus of course depends on the proposed implementation.  One such implementation (based on the work of Roberts (1996) and developed by B\"{u}ring (XXX) and Glanzberg (XXX)) proposes viewing discourse as a guided inquiry in which all contributions are part of a series of nested questions and answers.  Key to this idea is that the goal of inquiry is to arrive at a correct answer to the overall guiding question of the inquiry, and that contributions to the inquiry ought to advance the discussion toward that goal.  Of course, not all assertions made in a discourse are in response to directly posed questions.  So this proposal asks us to posit implicit questions to which these assertions are answers.  The result is an abstract conversational scoreboard that lists a set of questions and answers structured in such a way to show the progress of the discourse toward its conversational goal.  With this tool available, we can make sense of the felicity of the response in \Last[b].  In essence, the conversational scoreboard really looks something like the following.

\ex. \a. Who showed up to the conference?
\b. Who showed up to which part?
\b. [Sally]$_F$ came for the [first day]$_F$.

Now, the response is given an adequate antecedent via \textit{which part}.  As it stands, this proposal is rather unconstrained.  If we can just posit implicit questions of any sort whereever we need them, the theory doesn't carry much explanatory potential.  But, we have certain pragmatic tools at our disposal to render the proposal more theoretically palatable.  For instance, in our current example, it is plausibly common knowledge that conferences are temporally extended events, and that attendance at one stage of the conference does not entail attendance at all stages.  In attempt to be helpful, the respondant to the posed question is giving at least a partial answer.  Felicity does not require a complete answer to the explicitly expressed question; all that is required is that the response adequately implement part of a strategy to answering the question.  In this case the implemented strategy involves passing through the more narrowly focussed subquestion of which parts of the conference were attended by whom.

\section{Discourse coherence}

Gundel has suggested that contrastive focus be viewed as a distinct phenomenon from the set of phenomena that give rise to the given/new division and association with focus (she consolidates these phenomena under the heading of informational focus).  One of the motivations for distinguishing these two kinds of focus is that while informational focus appears to be linked to a mandatory feature of natural language in which semantic subject and predicate are distinguished, contrastive focus is not evident in all natural language sentences.  For Gundel, contrastive focus is simply a linguisitc realization of the more general tool of drawing attention to previously unnoticed subtleties of the conversational context.  With such a limited and pragmatic role to play, contrastive focus need not raise too much theoretical concern.

On the other hand, as brought out by the discourse based model of focus interpretation, one significant role that contrastive focus plays is in highlighting minor deviations from the expected structure of the discourse.  While it may not contribute to the truth conditional interpretation of sentences, contrastive focus is relevant to discourse coherence.  

\subsection{Focus and anaphora}

Focus clearly has a backward looking component.  One popular suggestion is that focusing gives rise to a presupposition, and in so doing requires an antecedent for that presupposition to be present in the preceding discussion.  Another popular suggestion is that focusing makes use of an anaphoric process where the antecedent is traced to the conversational context (modeled as a set of possible worlds or propositions, a la Kratzer).

There are three relevantly different approaches here:
\begin{itemize}
 \item Presupposition and explicit antecedent:  On this approach, focusing gives rise to a presupposition, which is realized explicitly at some level of interpretation.  The prior discourse is searched for an antecedent for that presupposition. (Geurts and van der Sandt)
 \item Tripartite structure and anaphora to propositions: On this approach, focusing is an anaphoric process.  The focus semantic value (an alternative set) must find an antecedent in the context set of the discourse.  Because the context set is a set of propositions (or the intersection of sets of worlds they determine), the anaphoric element must be such a set, too.
 \item Question underdiscussion discourse structuring:  On this approach, discourses are structured into nested sets of question/answer pairs, and focusing marks the strategy the speaker is taking within that structure.
\end{itemize}

\end{document}
