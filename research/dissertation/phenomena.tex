\documentclass[letterpaper]{article}
\usepackage[T1]{fontenc}
\usepackage[utf8]{inputenc}
\usepackage{lmodern}

%Symbol packages
\usepackage{amssymb,mathabx,stmaryrd}

%Margin settings
\usepackage{geometry}
\geometry{hmargin={.75in,.75in},vmargin={1in,1in}}

%Header settings
\usepackage{fancyhdr}
%\setlength{\headheight}{15pt}
\pagestyle{fancy}
\fancyhead{}
\fancyhead[L]{Erik Hoversten}
\fancyhead[C]{Interesting phenomena}
\fancyhead[R]{}

%Paragraph settings
\setlength{\parindent}{0pt}
\setlength{\parskip}{2ex plus .5ex minus .2ex}

%numbering and reference settings
\usepackage{hyperref}
\usepackage{linguex}
\usepackage{enumitem}

\title{}
\author{}

\begin{document}

%\maketitle

%\begin{abstract}
%\end{abstract}

\section{Interesting phenomena}

\paragraph{Dretske's contrastive statements} Clyde stands to inherit a large fortune if he gets married before he turns 30. He's been seeing Bertha off and on and enjoys the lax relationship.

\ex. \a. Clyde's reason for marrying [Bertha]$_f$ was to get the inheritance.
\b. Clyde's reason for [marrying]$_f$ Bertha was to get the inheritance.

\paragraph{Schaffer's contrastive knowledge ascriptions} Mary is an amateur birdwatcher, but doesn't have a lot of experience.

\ex. Mary knows that there's a canary in the backyard.
\a. QUD $= \{$there's a canary in the backyard, there's a crow in the backyard$\}$
\b. QUD $= \{$there's a canary in the backyard, there's a goldfinch in the backyard$\}$

\paragraph{Definite descriptions} The catastrophic nature of the existence presupposition varies from predicate to predicate.

\ex. \a. The king of France is bald.
\b. The king of France is sitting in this chair.

An information structure based account: One role of focus is to shift the standard semantic subject from the syntactic subject to the focused element. This allows for distinct readings in embeddings.  But intonational stress may not be the only way to shift semantic subjecthood.  In fact, world knowledge about particular predicates may be sufficient to allow those predicates to wrest subjecthood from the hands of certain syntactic subjects.  This is what I think is going on in examples like \Last[b] above.  The predicate demands attention, and the presupposition associated with the description gives way.

Yablo on `say more' presuppositions.  Presuppositions force us to ask what more is needed than the presupposition to make the sentence true.  In this way the presupposition helps us determine which proposition is expressed by the sentence, and they can play that role independently of their own truth.  And if the determined proposition is clearly true or false independently of the presupposition, then we may still be able to interpret it as such.  This idea also works as an explication of the Mill/Frege debate over the content of names.  The sense of names may usually be important for helping determine the referent, but once determined, we don't need the sense to do much interpretive work.  So sense is presuppositional.  $X$ bridges the gap between $\pi$ and $S$ iff $X$ is a $\pi$-free extension of $S$.

\paragraph{Contrastive focus} Intonational stress can carry a corrective import, which is a kind of denial and replacement content.

\ex. \a. Bears are dangerous.  No, [grizzly]$_{cf}$ bears are dangerous.
\b. I love bacon.  No, [everyone]$_{cf}$ loves bacon.
\c. Big Papi is pretty good at baseball.  No, Big Papi is [amazing]$_{cf}$ at baseball.

\paragraph{Goodman's counterfactuals} Logically equivalent antecedents can license very different consequents.

\ex. \a. If NYC was in Georgia, NYC would be in the south.
\b. If Georgia included NYC, Georgia would not be entirely in the south.

\end{document}
