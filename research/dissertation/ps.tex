\documentclass[letterpaper,]{article}

\usepackage[T1]{fontenc}
\usepackage{lmodern}
\usepackage{amssymb,amsmath}
\usepackage{ifxetex,ifluatex}
\usepackage{fixltx2e} % provides \textsubscript

\usepackage{geometry}
\usepackage{enumerate,verbatim,color}
\usepackage{fancyhdr}

%Margin settings
\geometry{hmargin={1in,1in},vmargin={1in,1in}}
\setlength{\parindent}{0.0in}
\setlength{\parskip}{3mm}
\setlength{\tabcolsep}{10pt}
\setlength{\arraycolsep}{10pt}

%Hyperref
\usepackage[colorlinks=true,linkcolor=blue]{hyperref}

%Biblio
\usepackage[backend=biber]{biblatex}
\bibliography{dissertation.bib}


\title{Correctible inquiry}
\author{Erik Hoversten}
\date{2015-01-12}

%Header settings
\usepackage{fancyhdr}
%\setlength{\headheight}{15pt}
\pagestyle{fancy}
\fancyhead{}
\fancyhead[L]{Erik Hoversten}
\fancyhead[C]{Correctible inquiry}
\fancyhead[R]{2015-01-12}

\begin{document}


\subsection{Levels of dynamism}\label{levels-of-dynamism}

Frege claimed that it was folly to ask for the meaning of a word in
isolation, instead intoning that it is the thought as a whole that is
the proper unit of semantic investigation. The dynamic turn in semantics
pointed to linguistic phenomena such as anaphora as evidence that
neither can single sentences be fully understood in isolation, thus
turning the gaze of semantic theorizing away from individual thoughts
and toward the broader linguistic context in which their specific
instantiations appear.

The many varieties of dynamic semantic accounts share in common a model
of discourses (be they conversations or texts) in which there is exists
a theoretical platform upon which the information of the discourse is
built in a stepwise manner, each subsequent contribution to the
discourse adding another level or wiping away previous construction.
This platform goes variously by names such as \emph{discourse
representation}, \emph{conversational scoreboard}, \emph{common ground},
and \emph{information state}. The materials of which discourses are
built include \emph{possible worlds}, \emph{situations}, \emph{events},
\emph{discourse referents}, and ordered or unorderered sets thereof.

These semantic theories are dynamic in that they describe how new
contributions to a discourse \emph{respond to} and \emph{modify} what
has preceded, and discourse is very much dynamic in this sense. But by
and large, they fail to capture another way in discourse is dynamic.
Contributions to discourse also \emph{anticipate} what is to follow in
the discussion, and this feature of how discourses proceed ought to be
brought under the umbrella of semantic theory.

\subsection{Collaboration and discourse
pairs}\label{collaboration-and-discourse-pairs}

Discourse is an essentially collaborative process. At a superficial
level, successful discourse requires that the participants accurately
understand each others' contributions. This requires constant feedback,
step retracing, and repair to keep everyone on the same page. But the
importance of collaboration goes deeper than this. Herb Clark and his
contributors (\textcite{clark1992}) have documented a variety of
discourse maneuvers in which an individual other than the interlocutor
who initiates the contribution provides essential input into its
completion. This has led Clark to suggest that discourse contribututions
involve two distinct phases: the \emph{initiation} and the
\emph{completion}. Crucially, in general, responsibility for the two
phases of contribution is distributed among the conversational
participants.

Clark is primarily concerned with the phonological and syntactic
representation of discourse contribtutions, and with how the
collaborative process determines the reference of elements of the
contribution. But the same insights extend to the level of
semantico-pragmatic representation as well.

A principle way in which the collaborative nature of discourse is
realized is in the way that assertions update the discourse context.
Following Robert Stalnaker (CITE: STALNAKER-ASSERTION), I understand the
principle effect of assertion to be the elimination of uncertainty,
which is modeled as the elimination of previously open possibilities
from the context set. But Stalnaker also recognizes that there are two
stages to the impact of an assertion. The first is an automatic update
that takes place implicitly. It involves incorporating information into
the common ground that includes things such as the fact that the speaker
is speaking, a record of the salient objects that have been introduced
to the context, and perhaps other forms of not-at-issue content of the
utterance. The second stage of assertion involves the restriction of the
context set to incorporate the at-issue content of the utterance. But
this second stage does not take place automatically. Instead, the
speaker's fellow interlocutors are given the power to either accept or
reject this proposal before the context set is updated. (It may be that
the proposal is entered into the conversational record in some form for
the purpose of anaphoric reference later on.) Thus, it is fitting to
think of assertion as involving two distinct phases: an initiation,
which involves the proposal of the asserted content by the speaker, and
a completion, which involves either the acceptance or rejection of the
asserted content by the hearer.

Another way in which the semantic level of discoruse representation is
collaborative involves questions and answers. Questions can of course be
accepted or rejected, but they also \emph{call for} a particular form of
response. That is, questions present an \emph{issue}, which carries with
it certain \emph{fellicity contitions} that acceptable responses must
meet, and the question cannot be considered closed until a response
meeting those conditions is proffered. We thus have what we can call a
\emph{discourse pair}: an ititiation, which is the issue proposed in the
question, and a completion, which is the answer. The initiation phase
presumes the possibility of a completion, and the completion depends
upon the issue proposed. And, crucially, the responsibility for the
elements of the pair is distributed among the conversational
participants. (Interestingly, this means that discourse initiations need
not be immediately followed by their completions. A questions may not be
(fully) resolved until after a number of clarificatory amendments or
sub-questions are introduced and resolved.)

Discourse pairs thus exhibit a recognizable pattern. A contribution is
an initiation if it presumes, or \emph{anticipates} a completion, which
is accomplished by means of exerting fellicity conditions on adequate
responses. A contribution is a completion if its content depends on, or
\emph{responds to} an initiation from prior discourse.

I think that the deeply collaborative nature of discourse suggests
another shift in semantic attention, expanding on the idea of discourse
dynamics, and treating discourse pairs as the primary units of semantic
evaluation. The novelty to this proposed shift is that semantic
evaluation must trespass not only the boundaries between sentences but
also the borders of turns taken in the conversational exchange.

\subsection{Posits and corrections}\label{posits-and-corrections}

The protagonist of the discussion to follow is what I've taken to
calling a \emph{posit}. Posits are speculative informational
profferments. In conversational exchange, an individual contributes a
posit when he puts forward for discussion information for which he does
not possess conclusive evidence.

Because they propose to add information to the common ground, posits are
similar to assertions. But posits, as I understand them, are not subject
to the same standards to which appropriate assertion is held.

In fact, posits can really only be understood in combination with their
natural discourse pair complement, the \emph{correction}. Since they
form a discourse pair, posits anticipate corrections, and a correction
is constrained to respond appropriately to a posit.

\subsection{The plan for the project}\label{the-plan-for-the-project}

The goal of this project is to motivate inclusion of posits within a
model of the structure of inquiry. The motivation has 5 parts:

\begin{enumerate}
\def\labelenumi{\arabic{enumi}.}
\item
  \hyperref[collab]{A development of a deeply collaborative model of
  discourse dynamics and an appeal for the inclusion of posits within
  this model via analogy to widely accepted elements of inquiry.}
\item
  \hyperref[hist]{A historical and conceptual motivation for the notion
  of speculation and correction in inquiry.}
\item
  \hyperref[formal]{A development of a formal model of posits and
  corrections within the update semantics tradition.}
\item
  \hyperref[contrast]{An application of the new proposal to an extant
  problem in discourse dynamics.}
\item
  \hyperref[relativism]{An application of the proposal to provide a
  novel understanding of a current debate in the philosophy of
  language.}
\end{enumerate}

\hyperdef{}{collab}{\subsection{Collaborative discourse dynamics and
types of contribution}\label{collab}}

Discourse between individuals who share a language can take a wide
variety of forms, but academic study has been primarily concerned with
the subclass of discourses that fall under the heading of cooperative
communicative exchanges of information. These discourses involve two or
more individuals engaged in a project of achieving a mutual goal
(generally, increased mutual knowledge), using eachother's stock of
information to increase their own. In this form, discourse implements a
jointly undertaken \emph{inquiry}. An inquiry is, in the words of Robert
Stalnaker, ``the enterprise of forming, testing, and revising beliefs''
\autocite*[ p.~ix]{stalnaker1987}.

While moves in an inquiry (discourse contributions) can be as varied as
the language allows, there is a small set of contribution types
derivable from the goals of inqury:

An individual can submit (propose) information to be taken on as
mutually accepted. We call this contribution an \emph{assertion}.

The end point of an inquiry depends in part on its starting point, which
is the \emph{issue} that is to be resolved by the process of inquiry. An
individual can submit a new issue to the inquiry to be taken on as
providing a new goal for the inquiry. We call this contribution a
\emph{question}.

Discourse is a joint endeavor, and each of these contribution types is
associated with an appropriate response from the contributor's
interlocutors. A contribution and its response constitute a
\emph{discourse pair}. For assertions, the appropriate response is
either acceptance or denial of the proposed addition to the common
ground. Call this a proposal/acceptance pair. For questions, acceptance
and denial are also important, which indicates that the question
provides a genuine addition to the common ground. But the question is
distinct from an assertion in that it calls out for an anwser, and the
form of the question constrains the set of appropriate answers. Call
this a question/answer pair.

Sometimes added to this list of contribution types is the
\emph{command}, which proposes a non-linguistic demand upon a
participant in the inquiry. The appropriate response to such a
contribution is to make the world such that the demand is met (or to
reject the demand itself).

Perhaps unsurprisingly, these contribution types each have an associated
linguistic mood. Inquiry is a structured process, in which it is
important not only what information is available but also how it is to
be put to use in reaching the goal. As an implementation of inquiry,
linguistic discourse has conventional means of representing how the
information is being put to use by a contribution. But the scope of
contribution types is not limited by the set of conventional moods. What
is important is that there is a recognizably unique way in which
information is put to use to further the goals of the inquiry.

One motivation for introducing a novel contribution type is by
classifying certain linguistic particles as carriers of that type. For
instance, a natural model of the semantic value of epistemic modals
(\autocite{fintel2010}, \autocite{veltman1996}) is that they provide a
test of the common ground. If the test is passed the entire common
ground passes through unaltered; if the test is failed, the common
ground collapses and a repair is required. But if this was all epistemic
modals did, their ubiquity in discourse would be mysterious, as their
function is reduced to trivial operations on the common ground. One way
of explaining the utility of epistemic modals in discourse is in terms
of their unique contribution type. This project has been undertaken
within the inquisitive semantics tradition, by accounting for the
contribution of epistemic modals as a way of highlighting particular
possible worlds within an information state
\autocite{groenendijk2014}.\footnote{This is, of course, not the only
  possible explanation of the unique value of epistemic modals. One
  could attempt to provide epistemic modals with a more nuanced semantic
  content, or couch their apparent additional contribution in terms of
  general pragmatic principles.} \emph{Attention shifting}, then, is a
novel contribution type because it makes use of information in a unique
way. And it, too, is associated with an appropriate response. The
appropriate response to such a contribution is to direct ones further
contributions to this highlighted piece of the total common
ground.\footnote{Additional contribution types may include
  \emph{suppositions}, which are carried by the antecedents of
  conditionals.}

These examples do not provide an exhaustive list of distinct
contribution types, nor do I suggest that I can provide identity
conditions specific to types of contribution. But these examples to
share certain things in common, and their common traits can provide us
with criteria for recognizing novel contribution types.

First, each of these types anticipates an appropriate response, and
places constraints on the form of the response. Second, each type
outlined above operates on information in a different way. The currency
of inquiry is information, and its value is tied to the ways in which
information can be put to use to advance the inquiry toward its goal.
And last, each of the types above is carried by a conventionalized
linguistic tool. If a contribution belongs to a specific type, language
will have found a way to implement that type.

Criteria for genuine contribution types:

\begin{enumerate}
\def\labelenumi{\arabic{enumi}.}
\itemsep1pt\parskip0pt\parsep0pt
\item
  A genuine contribution type has a natural discourse pair complement.
\item
  A genuine contribution type implements a unique operation on
  information.
\item
  A genuine contribution type is realized by a conventionalized
  linguistic tool.
\end{enumerate}

I think that posits constitute a genuine contribution type distinct from
the others listed. Posits are similar to assertions in that they seems
to provide a proposal of added information, and they are similar to
questions in that they seems to present an issue for discussion. They
are also similar to suppositions in that they seems to be less than
fully committal on the part of the speaker.\footnote{Assertion is
  generally associated with a commitment that it generates for the
  speaker. This commitment involves some evidential relation between the
  asserter and the information conveyed, though it is controversial just
  what relation it is. But whatever the relation amounts to, it seems
  clear that we sometimes offer contributions to discourse that go
  beyond the evidence we have available. And we do this not just as a
  means of flouting the rules of discourse, but frequently in order to
  respect the project of the communicative exchange.} But posits are
distinct from each of these, and call for a unique treatment. That they
earn such status is what I hope to show throughout the course of this
project.

\hyperdef{}{hist}{\subsection{Speculation and correction in inquiry:
historical perspectives}\label{hist}}

\subsubsection{William James and the will to believe}\label{james}

William James famously drew a distinction between two independent and
sometimes conflicting goals for those engaged in the pursuit of
knowledge. The first is to acquire true beliefs. The second is to avoid
believing falsehoods. One could meet the goal of believing truths simply
by believing everything, but doing so sacrifices entirely achieving the
second goal. And one could meet the goal of avoiding falsehoods by
believing nothing at all, thereby forfeiting the virtue of believing
truths. As James saw it, a strategy of inquiry that floats between these
extremes is the path we ought to search for.

James admits that the second goal, that of avoiding false belief, seems
to carry more sway, especially in the realm of scientific inquiry.
Belief is conceptually tied to action, and while acting on false belief
is frought with peril, witholding belief and failing to act due more
rarely leads to trouble. But sometimes non-action is itself action, and
peril can find those who remain in one place as well as those who run
into its path. James' project was to characterize the situations in
which inquiry optimization favors the goal of truth acquisition, even at
the potential expense of error avoidance. His proposal, famously, was
that for situations in which one's belief choice is \emph{live},
\emph{forced}, and \emph{momentous}, the will to believe outstrips the
fear of being wrong. Unfortunately, this category leaves out most
quotidian inquiries; perhaps some people ponder the existence of God on
a daily basis, but most of us are happy just to complete our grocery
shopping all in one trip.

Even James, the arch supported of faith-based belief, has restricted his
spectulative endeavors too far. Inquiry need not be momentous or forced
for a jump to a conclusion to be fruitful. James' model makes inquiry an
optimization task between taking on too many falsehoods and leaving out
too many truths. And it is a task that is undertaken by fallible
individuals in real time. Such agents have limited access to information
and limited resources to dedicate to the process of inquiry
\autocite{bratman1988}. Frequently, action is required when certainty
cannot be obtained. And on many quotidian decision points, the risk of
being incorrect is fairly low. It is \emph{because} not every inquiry is
momentous that striving for truth (and risking falsehood) is sometimes
to be prefered.

A preference for belief in the face of uncertainty is further supported
by the fact that inquiries are packaged with corrective feedback
mechanisms. An important such mechanism is the build and test update
procedure that inquiry exhibits. Inquiries are cummulatively built in a
step by step process. One contributor to the inquiry adds a proposal to
the mutual inquiry workspace at which point it can be played around with
by all members of the inquiry before being ultimately accepted or
rejected.

These corrective mechanisms are an important reason that efficiency
often demands that we make speculative additions to the common ground as
opposed to waiting until we can obtain certainty that the addition is
true (or otherwise appropriate). But the success of this procedure
depends highly on a shared understanding of each party's role in the
inquiry. An individual can only feel comfortable in proposing a
speculative addition to the common ground if they believe that their
collaborators will correct their contribution to the best of their
ability. And this requires interlocutors to be more than passive
recipients of information. They must use openings in the conversation to
voice disapproval or uncertainty in addition to registering
understanding and acceptance of what has been presented.

Disagreement is the tool by which interlocutors can check each other's
flights of fancy. And, I maintain, it is a tool we both know how to
wield, and are prepared to let others weild against us. The primary role
of joint inquiry is to expedite what would be an extremely tedious task
if attempted alone. Even alone, the task can be sped up via judicious
application of speculative intellectual leaps, though the risk involved
cannot be wholly eliminated. The collaborative facet of joint inquiry
provides extra motivation for assuming the risk -- the cooperative
interlocutor, to the extent she is able, will pull you back from the
ledge by voicing her disagreement.

David Lewis \autocite*{Lewis1975b} spoke of language as a convention of
truth and trust in a community. His idea was that our shared language is
supported by a tacit agreement to speak truthfully, and to trust that
others will do so as well. While the Jamesian model accepts the
importance of speaking truthfully, it maintains that the convention may
sometimes call for speaking rather than remaining silent even when truth
cannot be guaranteed. The importance of trust remains paramount, but it
is not just the hearers' trust in the speaker to utter truly. It is also
the speaker's trust in his hearers to correct him when he doesn't.

\subsubsection{J. S. Mill}\label{mill}

John Stuart Mill's \autocite*{mill1859} defense of liberty similarly
centrally relies on the corrective value of dispute. Even in the face of
certainty regarding the answer to certain questions, dispute plays an
important role. To be always ready to dispute opponents' claims keeps
one's defense of their own claims fresh.

In successful inquiry, it is not enough that the participants in the
inquiry come to accept truths about the world. The process of inquiry,
when faithfully undertaken serves to provide its participants with
justification for coming to accept the outputs of the inquiry. Having
gone through the inquiry is also necessary for the right to believe its
outputs.

But this makes one's right to believe dependent upon the interlocutors
with whom one interacts. Not only are asserters expected to be able to
provide reasons for the propositions they propose, hearers are expected
to challenge assertions when their content contradicts either their own
expectations, or the shared conversational record.

These commitments of cooperative exchange are realized in certain
discourse mechanisms. The hearer has certain tools at her disposal for
challenging assertion. Primary among these is flat out denial, but often
it is not maximally cooperative to flatly deny an assertion for this
response has the effect of derailing the progress of the inquiry, and it
is too sweeping in its effect. Utterances involve a great number of
parts, any of which may be subject to criticism, but criticizing one
part may leave others unscathed. Thus, hearers also have means for
correcting utterances without issues a blanket denial.

\subsubsection{C. S. Peirce}\label{peirce}

For C. S. Peirce, scientific inquiry involved 3 distinct components,
each of which provides a distinctive justification for the outputs of
the process, and each of which is subject to a distinctive form of
evaluation. An initial application of the abductive phase generates a
set of hypotheses which serve as potential answers to the posed
scientific issue. This phase is the least regulated of the components of
scientific inquiry. Peirce maintained that there are rules governing
proper application of creative abduction, but many philosophers of
science have maintained that analysis of this process is limited to the
psychology of scientists. A deductive phase generates necessary
consequences of the hypotheses generated in the first phase. The rules
of deductive logic govern the outputs of this phase. In inductive phase
then steps in to test the hypotheses by comparing their deductive
commitments to the outputs of empirical testing. And finally, a phase of
selective abduction uses the results of the inductive phase to select
the best hypothesis for whatever purposes such a selection is needed:
further testing, belief updating, or being proclaimed as scientific
fact.

If Peirce's suggestion that these components represent general processes
of human reasoning, then we would suspect that inquiry generally
manifests the same set of steps. As an inquiry in its own right,
discourse, too, should exhibit each of these components.

\hyperdef{}{formal}{\subsection{Incorporating posits and corrections
within update semantics}\label{formal}}

I situate the formal implementation of my proposal within the class of
theories known as \emph{update semantics}. The key feature setting
update semantics apart from classical static semantics is that the
semantic values of meaningful linguistic particles (or mental entities)
are not worldly objects to which those particles \emph{refer}. Instead,
semantic values characterize the way in which a \emph{state of
information} is changed in response to the particles.\footnote{\emph{Update}
  semantics further differ from merely \emph{dynamic} semantics, in that
  states of information are built up with each new update as opposed to
  being erased and rewritten (cf.
  \autocite{dekker-pla}, \autocite{groenendijk1991}).}

I follow Sarah Murray \autocite*{murray2014} in proposing that updates
fall into one of three broad categories based on the nature of the
change they bring to the state of information. \emph{Direct updates}
alter the makeup of the state of information, principally by eliminating
elements thereof; \emph{structural updates} alter the relations holding
between elements of the state of information, perhaps by instituing an
ordering among them; and \emph{dref updates} introduce a new element to
the state of information, the principle purpose of which is to make them
available for anaphoroic reference.\footnote{It's possible that these
  update types are not mutually exclusive. It's also possible that
  theories that differ only in notational ways may assign updates to
  different types. For instance, one may take an update to directly
  alter an element of the common ground to reflect a relation among
  \emph{its} elements, in which case we would consider it to be a
  structuring update. Or, one could add a specification of a domain and
  range as a new element of the state of information, thereby providing
  all the tools necessary for constructing the relation among elements
  of another element of the state of information. In this case, the
  update appears to be of the dref variety. In effect, the first option
  replaces one set of entities with a \emph{relation-in-extension} among
  the same entities, and the second option adds a
  \emph{relation-in-intension} to the information state. I don't think
  there is reason to bicker over the details here so long as both
  options equally account for intuitive interpretations of the
  linguistic item under consideration. However, there may be empirical
  reasons to choose one over the other. If the update functions, for
  instance, by replacing an unordered set by and ordered one, then we
  may take certain information to be lost; namely, that the state once
  represented the set as unordered. But if the relation-in-intension is
  added to the information sate, then we have the tools to construct the
  relation-in-extension without directly doctoring the set to which it
  applies. It may be that adequte interpretation of certain extended
  discourses requires the presence of the original, undoctored set in
  addition to the relation, but I don't investigate this idea here.}

Posits involve both direct and structural updates. Understanding why
this is requires making sense of their role within the context of a
posit/correction pair. It also requires addressing two key properties of
of the use of posits in discourse.

\begin{enumerate}
\def\labelenumi{\arabic{enumi}.}
\item
  Posits have \emph{disputative potential}. The use of a posit calls out
  for correction if an interlocutor can provide it. Corrections can
  address a wide variety of features of an utterance, and an account of
  the semantic value of posits must make these features available for
  potential correction. I account for this feature by making use of the
  fact that within an update theoretic model, individual utterances (or
  even portions theoreof) can contribute a multitude of independent
  updates to the state of information. Corrections can address any of
  these many updates without impugning the others.\footnote{Since, in an
    update theoretic model, a semantic value need not be linked to a
    distinct worldly entity, sentences (or thoughts) can be semantically
    complete and meaningful without the need to posit some thing that
    the sentence (thought) picks out. Instead, semantic values of
    complex meaningful particles can be treated as complexes of semantic
    values, each of which changes the state of information in its own
    way, and sequentially.}
\item
  Posits are \emph{resiliant}. The use of a posit need not be completely
  withdrawn in the face of disagreement. Posits update the common
  ground, but they don't seem to carry the level of commitment that is
  generally associated with assertions. This property is captured in
  part by the previously noted fact that even when corrected, utterances
  make multiple updates that pass unscathed. But this doesn't fully
  capture the power of making posits over assertions. To account for
  this, I introduce the notion of a \emph{default saturation} of
  property level updates.\footnote{Also important to capturing posit
    resiliance is the idea that discourse pairs constitute the basic
    unit of semantic evaluation.}
\end{enumerate}

Along the way, we address issues such as to what semantic type
(\emph{e}, \emph{et}, \emph{s}, etc.) posits belong, on what dimension
(semantic, conventional implicature, presupposition) the interpretive
import of posits ought to be placed, and against what domain to
interpret default saturation.\footnote{Comparison to inquisitive
  semantics?}

\hyperdef{}{contrast}{\subsection{The discourse contribution of
contrastive topic}\label{contrast}}

Focus is a lingusitic tool whose use serves primarily to \emph{package}
information as opposed to providing its own contribution. It
distinguishes the elements of the sentence in which it occurs that are
new to the discussion from those that have already been introduced, thus
showing to the other participants in the discourse how the contribution
is intended to fit in with what has preceeded. In this sense, focus has
a distinctly \emph{backward looking} function. It depends on, and
responds to, previous moves in the conversation. So it makes sense that
the most common analyses of focus interpretation asign it a
\emph{presuppositional} semantic function. Focus adds a \emph{fellicity
condition} to the overall import of a sentence; the sentence is
interpretable in situ only if material matching the focus semantic value
of the sentence can be found in the previously constructed conversation.

The star bit of data that is marshalled in support of this proposal is
that of question/answer congruence. Compare the responses to the
following question:

\begin{enumerate}
\def\labelenumi{(\arabic{enumi})}
\itemsep1pt\parskip0pt\parsep0pt
\item
  Who ate the last cookie?

  \begin{itemize}
  \itemsep1pt\parskip0pt\parsep0pt
  \item
    {[}Caitlin{]}f ate the last cookie.
  \item
    Caitlin ate the last {[}cookie{]}f
  \end{itemize}
\end{enumerate}

While response B is a perfectly natural answer, the placement of focus
in C makes it stand out as unacceptable. According to a popular analysis
of questions, due to Hamblin, the semantic value of a question is the
set of propositions that count as answers to it. So, a (suitably
contextually constrained) semantic value for question A might be:

\begin{enumerate}
\def\labelenumi{(\arabic{enumi})}
\setcounter{enumi}{1}
\itemsep1pt\parskip0pt\parsep0pt
\item
  \{\emph{Albert ate the last cookie}, \emph{Betsy ate the last cookie},
  \emph{Caitlin ate the last cookie}, \ldots{}\}
\end{enumerate}

And, the proposed analysis of focus renders the following focus semantic
values for the given responses:

\begin{enumerate}
\def\labelenumi{(\arabic{enumi})}
\setcounter{enumi}{2}
\item
  \{\emph{Albert ate the last cookie}, \emph{Betsy ate the last cookie},
  \emph{Caitlin ate the last cookie}, \ldots{}\}
\item
  \{\emph{Caitlin ate the last apple}, \emph{Caitlin ate the last beet},
  \emph{Caitlin ate the last cookie}, \ldots{}\}
\end{enumerate}

The alternative set generated by the focus in B is the same as the
semantic value for question A, and that generated by the focus in C is
not. Add in the proposed fellicity condition, and you have a great
explanation of question/answer congruence.

Analyzing focus interpretation in terms of a pragmatic rule connecting
focus semantic values to semantic antecedents in prior discourse is
incredibly fruitful. But now consider the following example:

\begin{enumerate}
\def\labelenumi{(\arabic{enumi})}
\setcounter{enumi}{4}
\item
  \begin{itemize}
  \itemsep1pt\parskip0pt\parsep0pt
  \item
    What did people bring to the picinic?
  \item
    The {[}children{]}\textsubscript{cf} brought {[}unbridled
    enthusiasm{]}\textsubscript{f}.
  \end{itemize}
\end{enumerate}

This response involves two focused elements. For both elements, it is
easy enough to generate the requisite alternative sets, and alternative
set generated for the second focused element (marked with a subscripted
\emph{f}), is perfectly congruent with the preceding question. But a
difficulty arises when we attempt to apply the fellicity condition to
the element labeled with \emph{cf} (for \emph{contrastive focus}). Its
alternative set is distinctly not congruent with the preceding question,
so we would expect the entire utterance to be marked as infellicitous.

\subsubsection{The problem of contrastive
focus}\label{the-problem-of-contrastive-focus}

But now consider the following discourse:

\begin{enumerate}
\def\labelenumi{(\arabic{enumi})}
\setcounter{enumi}{5}
\item
  \begin{itemize}
  \itemsep1pt\parskip0pt\parsep0pt
  \item
    Bears are dangerous.
  \item
    {[}Grizzly{]}\textsubscript{cf} bears are dangerous.
  \end{itemize}
\end{enumerate}

The response in the example above exhibits what is often called
contrastive focus. While there are some differences in the intonational
patters between the focus mentioned earlier and contrastive focus
(cf.~Gundel), there is reason to think that much the same mechanism is
at work in the interpretational effects of both. But this example poses
a challenge for the proposed analysis of focus interpretation. There is
no trouble generating an alternative set for the focused element. The
issue is that there is no explicit antecedent for the generated set.

Daniel Buring has supplied us with a technique to locate the requisite
antecedent. Building on the notion of the question under discussion
developed by Cragie Roberts, Buring posits the existence of a discourse
tree, which provides structure for the inquiry being undertaken. Each
question governing the inqury is a node in the tree and is associated
with an array of subquestions, complete answers to which constitute
partial answers to the parent question. The idea is that the
subquestions provide a \emph{plan of inquiry} for addressing the parent
question. The objective of the inquiry is to answer the top-most
question in the tree. The plan for doing so is to answer each of the
subquestions in turn.

But how should we apply this porposal to the current case, where
contrastive focus is present without any explicit question? We can, of
course, appeal again to the ever present implicit question under
discussion. All discourses, it would be maintained, whether questions
have been explicitly introduced or not, are governed by a plan of
inquiry, which is structured by questions, implicit if need be. Then,
the explanation of the contrastive topic in (6) would be that the
original assertion offers an answer to a parent question, and the
correction that follows rejects this answer, replacing it with an answer
to a subquestion.

This proposal certainly fits with the structure of the question under
discussion representation of the conversational scoreboard. But we may
be left a little unsatisfied with the details. First, it seems to be
straining the notion of presupposition pretty heavily. Not only is there
no direct representation to stand as antecedent to the generated focus
semantic value, {[}AS WAS POINTED OUT EARLIER{]} contrastive topic is
quite resiliant to infellicity. Many different potential elements could
have been fellicitously focused, and it's hard to see how the
conversational scoreboard could have all of these options ready to hand.
The common response to this sort of worry is to appeal to accommodation.
If the scoreboard does not contain the d-tree needed to provide the
antecedent for the contrastive focus, then straightaway it comes about.
This reliance on accommodation is not without its own set of issues,
predominantly stemming from the unconstrained character of accommodation
as a theoretical posit.

But what I take to be the central issue with this sort of approach to
the problem of contrastive topic in corrections is a conceptual one
having to do with the role of corrections in discourse evolution. The
question under discussion model of contrastive topic makes the discourse
contribution of contrastive topic entirely backward looking. Whatever
role the focus plays is just a matter of checking prior discourse for an
antecedent to the focus generated alternative set.

\hyperdef{}{relativism}{\subsection{Disagreement and the relativism v.
contextualism debate}\label{relativism}}

Disagreement has been a bit of a star figure in recent debates in the
philosophy of language. Primarily, it has been used as a tool for
assessing the adequacy of different proposals for the semantic
contribution of certain expressions. Genuine disagreements, as opposed
to spurious or merely apparent ones, intuitively require in the
linguistic context or the minds of the disputants, the presence of some
kind of content toward which the participants have incompatible
commitments.

The contestants in battle disagreement are \emph{absolutism},
\emph{contextualism}, and \emph{relativism}, and the standard scorecard
has absolutism and relativism coming out on top in virtue of their
ability to secure the requisite constant content across uses of the
expression. Contextualism, it is charged, assigns contents to the
utterances and thoughts of different individuals that have them
incurring different, compatible commitments as a result of their
utterances or thoughts.

Whatever stance we take on the outcome of this debate, disagreement does
seem to be a valuable tool to have in the linguist's toolkit. Language
users have intuitions about the compatibility of sentences used in
discourse, and linguistic theories ought to respect those intutions. But
simply marking the presence of incompatible contents in a discourse is a
relatively peripheral role for disagreement to play in lingusitic
theorizing. As they stand, the semantic theories considered above give
no insight into disputative discourse itself; they merely accord to a
greater or lesser extent with one interesting consequence of such
discourse.

The previous discussion has shown that, in at least one way,
disagreement plays a significant role in the very structure of
communicative exchange; namely, in the posit/correction discourse pair.
If the model provided is sound, it provides us with a different avenue
for exploring the importance of disagreement as linguistic data. And, I
believe, following the data where it leads provides us with a new
motivation for incorporating relativism in semantic theory.

\subsection{References}\label{references}

\printbibliography

\end{document}

