\documentclass[letterpaper,10pt]{article}

\usepackage[colorlinks=true,linkcolor=blue]{hyperref}
\usepackage{amssymb}

%Margin settings
\usepackage{geometry}
\geometry{hmargin={.75in,.75in},vmargin={1in,1in}}

%Header settings
\usepackage{fancyhdr}
%\setlength{\headheight}{15pt}
\pagestyle{fancy}
\fancyhead{}
\fancyhead[L]{}
\fancyhead[C]{}
\fancyhead[R]{}

%Paragraph settings
\setlength{\parindent}{0pt}
\setlength{\parskip}{2ex plus .5ex minus .2ex}

%numbering and reference settings
\usepackage{hyperref}
\usepackage{linguex}

\begin{document}

\section{Inquiry based motivation for corrective potential}

The category of discourse that I am interested in is that kind that be fruitfully described teleologically as shared pursuit of knowledge.  Discourses of this category are often labeled inquiries. In regard to the aim of knowledge, we should keep in mind William James' point that an agent with this aim has two sometimes conflicting agenda: to believe truths and to avoid believing falsehoods. Discourses of this kind are really only of interest to agents with limited resources.  I take it that inquiry also proceeds with these competing agenda tugging at the reins.  The context of limited epistemic resources within which all interesting inquiries take place, pushes interlocutors into leaps of fancy, with the hope that an attentive interlocutor will correct them to the extent that they can.

Much concern in semantics was on truth conditional content.
This advanced to develop better accounts of content for non-assertoric speech acts, such as questioning and demanding.
We added accounts of cross-sentential relations, such as anaphora.
And assertions that say as much about an agents epistemic state as about the world itself have been pursued.
Parallel effets of utterances, such as implicature, presupposition, evidentials, appositives, relative clauses, metalinguistic negation and more have received treatment.
Seemingly relative domains, in which content appears to be determined by factors outside the speakers purview, have been made less mysterious.

And we have even examined sentences whose primary purpose is disagreement.  Though, this discussion has primarily developed with those sentences as data points. But I think that disagreement is more than just a reactive attitude.  It plays a crucial role in any fruitful inquiry.  Do such an extent that it may be worthwhile to rethink the semantic nature of sentences in terms of the disagreement they may give rise to.  

That is, an additional interesting feature of sentences is their \textit{corrective potential}.

Consider the following discourse:

\ex. \a. Bears are dangerous.
\b. Well, [Grizzly bears]$_{cf}$ are dangerous.

The response in \Last[b] should be understand as a rejection of the assertion in \Last[a].  But notice that it is no flat footed denial.  It instead, offers an alternative proposal while remaining perfectly on topic.  Let's call this a \textit{corrective assertion}. I take it that these sorts of conversational moves are ubiquitous.  And I suggest that they are so because of a central role they play in the progression of inquiry.  But we shouldn't ignore the speaker of \Last[a]'s role either.  She provides the perfect alley to B's oop of informational score.  My suggestion is that A has served the inquiry with an assertion that stabs at truth, and B has taken up the lead, honing the information only slightly.

If this story is correct, then the corrective potential of A is something to be represented in a formalized account of its interpretational import. 

This provides a motivation against simply treating B's assertion as a denial plus a new assertion with a more restrictive content.
And we also want to distinguish this position from a purely presuppositional account of focus, which would say that there exists a strategy to answer the question under discussion which matches the alternative set derived from the contrastively focused material.

\section{Corrective potential examples}

\paragraph{Homogeneity presupposition} The use of plural referring terms as well as predicates carries with it the idea that the individuals quantified over are uniform with regard to the predication.  This presupposition gives rise to warranted corrections such as in \Last.

\paragraph{Completeness presupposition} This is related to Grice's informativity maxim that one should say the strongest thing they are in the position to say. The presupposition gives rise to warranted corrections such as:

\paragraph{Uniqueness presupposition}

\paragraph{Disagreement}  Disagreements arise from challenged corrections.  Contextualism has low correction potential.  Relativism captures correction potential, which is why it may serve to explain what is going on in contrastive focus.

\end{document}
