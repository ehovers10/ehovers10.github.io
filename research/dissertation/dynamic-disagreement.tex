\documentclass[letterpaper,10pt]{article}
\usepackage[T1]{fontenc}
\usepackage[utf8]{inputenc}
\usepackage{lmodern}

\usepackage[colorlinks=true,linkcolor=blue]{hyperref}
\usepackage{mdwlist,verbatim}

%Margin settings
\usepackage{geometry}
\geometry{hmargin={.75in,.75in},vmargin={1in,1in}}

\title{Dynamic disagreement}
\author{Erik Hoversten}

\begin{document}

\maketitle
%\tableofcontents

\begin{abstract}
\end{abstract}

\section{Disputes and inquiry}

Disputes can be frivolous, fractured, and fraudulent. They can derail conversations, and generate so much animosity that perpetuating the dispute itself comes to mean more than the prospect of resolving it.  These disputes may have very little to do with actual conflict over the informational potential of the disputative language used to express them, instead being driven by \textit{ad hominem} and \textit{ad ideologiam} tendencies.

But dispute can also be a fruitful tool for people involved in an honest, cooperative attempt to further mutual knowledge. There is the ancillary fact that interlocutors sometimes lose track of the cooperative effort, and it is valuable to have a means of correcting them to bring them back on to the task at hand.  But dispute can play also more central role in the project of inquiry.  

One argument for this picture of the role of dispute in inquiry stems from a Jamesian picture of the strive for knowledge. It is a two-sided endeavor of acquiring true beliefs and avoiding false ones.  One could meet the goal of believing truths simply by believing everything, and one could meet the goal of avoiding falsehoods by believing nothing at all.  To optimize on both goals requires a strategy of belief that falls between those two extremes.  Sometimes advancing one's knowledge requires believing in ways that go beyond one's evidence.  Doing so opens avenues of investigation, and corrections to belief can be made after investigation closes.  The same goes for joint inquiry.  One can cooperatively assert what may not be certain for the purpose of advancing the inquiry along a particular line.  Given that this is a recognized cooperative manuever, dispute is a crucial tool for interlocutors to correct faulty speculations.  From this perspective, inquiry is not just a series of related but independent assertions that sequentially build the common stock of information.  Instead, it is a deeply interactive process of speculative advancements and corrective retreats.\footnote{In so far as such proposals put forward in inquiry are considered to be assertions, this picture is in direct conflict with a view of assertion as controlled by a norm of knowledge. (cf. Williamson (2000))}

An alternative argument for the same picture stems from Millian considerations of the virtue of freedom of intellectual expression.  To be always ready to dispute claims keeps one's defense of their own claims fresh.

My goal in the following is to take this view of dispute as central to inquiry as a jumping off point.  I wish to argue that the contemporary debate over the role of disagreeemnt in the philosophy of language can be improved by giving dispute this central role, and that doing so motivates viewing disagreement in a dynamci light.  

From here, I look at a particular linguistic phenomenon that seems to me to provide an interesting test case in the function of dispute in language use and interpretation.  

\section{Jamesian inquiry}

William James famously drew a distinction between two independent and sometimes conflicting goals for those engaged in the pursuit of knowledge.  The first is to acquire true beliefs.  The second is to avoid believing falsehoods.  One could meet the goal of believing truths simply by believing everything, but doing so sacrifices entirely the second goal.  And one could meet the goal of avoiding falsehoods by believing nothing at all, thereby forfeiting the virtue of believing truths. As James saw it, a strategy of inquiry that floats between these extremes is the path we ought to search for.

James admits that the second goal, that of avoiding false belief, seems to carry more sway, especially in the realm of scientific inquiry.  Belief is conceptually tied to action, and while acting on false belief is frought with peril, failing to act due to failure to believe more rarely leads to trouble.  But sometimes non-action is itself action, and peril can find those who remain in one place as well as those who run into its path.  James' project was to characterize the situations in which inquiry optimization favors the goal of truth acquisition, even at the potential expense of error avoidance.  His proposal, famously, was that for situations in which one's belief choice is \textit{live}, \textit{forced}, and \textit{momentous}, the will to believe outstrips the fear of being wrong. Unfortunately, this category leaves out most quotidian inquiries.  Perhaps some people ponder the existence of God on a daily basis, but most of us are happy just to complete our grocery shopping all in one trip.

So, it's not clear that the Jamesian model can be applied to inquiry generally.  In so far as the goal of inquiry is ultimately knowledgeable action, it may be better to take the conservative approach of only modifying the shared information state in the presence of certainty. Still, the interpretation of ``live'', ``momentous'' and ``forced'' all depend on the relevant context.  Few, if any, decisions are forced \textit{simpliciter}, but many seem unavoidable given simple background considerations.  I don't have to pick up either seitan or tofu from the store, but given certain desires I have -- to get protein without harming too many folks along the way, I've got to make up my mind before the store closes.  Perhaps, then, James' project can be pulled off, even for everyday inquiry.

But the adventurous inquirer need not accept context's life preserver.  Even James, the arch supported of faith-based belief, has restricted his spectulative endeavors too far.  Inquiry need not be momentous or forced for the jump to conclusions to be fruitful.  James' model makes inquiry an optimization task, between taking on too many falsehoods and leaving out too many truths.  And it is a task that is undertaken by fallible individuals in real time.  It's because not every inquiry is momentous that striving for truth (and risking falsehood) is sometimes to be prefered. Inquiries can withstand failures.  This is in part because not every choice is a matter of life or death, but also because inquiries are packaged with corrective feedback mechanisms.  The principle  one of which is the inquirer's fellow interlocutors.

Disagreement is the tool by which interlocutors can check each other's flights of fancy.  And, I maintain, it is a tool we both know how to wield, and are prepared to let others weild against us.  The primary role of joint inquiry is to expedite what would be an extremely tedious task if attempted alone.  Even alone, the task can be sped up via judicious application of speculative intellectual leaps; though the risk involved cannot be wholly eliminated.  The collaborative facet of joint inquiry provides extra motivation for assuming the risk -- the cooperative interlocutor, to the extent she is able, will pull you back from the ledge by voicing her disagreement. 

David Lewis spoke of language as a convention of truth and trust in a community.  His idea was that our shared language is supported by a tacit agreement to speak truthfully, and to trust that others will do so as well.  While the Jamesian accepts the importance of speaking truthfully, she maintains that the convention may sometimes call for speaking rather than remaining silent even when truth cannot be guaranteed.  The importance of trust remains paramount, but it is not just the hearers' trust in the speaker to utter truly.  It is also the speaker's trust in his hearer's to correct him when he doesn't.

\section{Millian inquiry}

John Stuart Mill famously maintained that liberty of opinion was to be maintained, even in the face of overwhelming evidence in favor of a single viewpoint.  To dismiss dissention is to calcify belief.  Solid belief supports confident action, which can be virtuous.  But often that confidence extends beyond its warranted scope, which is surely a vice.  Even when inquiry has seemingly reached its end, active disagreement nourishes subsequent inquiries.

Disagreement, then, serves a vital role in the process of inquiry.  In addition to correcting the current inquiry when it begins to slide off the rails, its nourishing effects extend to future inquiry and strengthen the skill of inquirers.

\section{Inquiry dynamics}

What I have been trying to motivate is the idea that disagreement is not merely an incidental feature of conversation, but a crucial element of healthy inquiry.  The role that disagreement plays in these models of inquiry also indicates an important feature.  Disagreement is not just a matter of conflicting discourse elements.  It is a step in the process of reaching the end of inquiry.  It is dynamic.

I suggest we ought to view disagreement not just as a relation between contents (as we would entailment or negation), but as an element of discourse dynamics (as we would assertion or updating generally).

Of course, we need not be pushed to this conclusion.  Disagreement has two parts.  There is the conflict, and there is the engagement.  We could treat these as separate elements deserving separate explanations.  

In a similar, but distinct context, Andy Egan has utilized such a proposal to attempt to account for the discourse dynamics of talk involving the de se.  The standard view of assertion, due to Stalnaker, has it that the result of (accepted) assertion is the intersection of the common ground with the asserted content. But this standard account yields unacceptable results when the de se is involved.  When I contribute ``I'm hungry'' to our conversation, I intend you to come to believe that Erik is hungry.  But if the content of ``I'm hungry'' is the property of being hungry, then for you to take on the state that results from intersecting our common ground with that content, you will come to believe that \textit{you} are hungry.

Egan's proposal is to limit the scope of de se contents in discourse dynamics.  When interlocutors are sufficiently similar in regard to properties relevant to the conversation at hand, it can be fruitful to assert de se contents.  It's just that most uses of sentences involving personal pronouns do not serve that function.

\section{Inquiry and knowledge}

Inquiry is a process with an optimal endstate that is knowledge.  But the individual steps of the process need not generate or preserve knowledge.  Inquiry is a distinctly non-monotonic process.  This is the essential problem with the knowledge account of inquiry.  

Ways in which we violate monotonicity:
\begin{itemize}
  \item Default reasoning.  Epistemic modals don't always result in state restriction.
  \item Questioning.  States can be partitioned without being restricted.
  \item Positing. We can put forward working hypotheses that go beyond our evidence.
  
\end{itemize}

\section{Two literatures}

There is a sizeable literature on using disagreement as a data in evaluating semantic proposals for various expressions.  The idea is that genuine disagreement can only take place in the presence of some sort of shared content, and that only certain semantic proposals provide that content.  The question then becomes whether certain classes of expressions are involved in instances of genuine disagreement.

There is also a literature surrounding the linguistic import of corrections.  Corrections are negative responses to utterances, which also provide updated information proposals.

\section{Wide content, wide attitudes}

Does allowing that an attitude may take wide content commit one to its always taking wide content?



\end{document}
